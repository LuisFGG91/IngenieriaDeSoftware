\section{Equipo}
\subsection{Williams Aránguiz - Product owner}
\begin{figure}[h!]
\centering
\includegraphics[width=7cm]{Figuras/Williams.png}
\end{figure}

En concreto, el \textit{Product Owner} procura que el equipo \textit{Scrum} aporte valor al negocio en cuestión. Él representa a los \textit{stakeholders} o a las partes interesadas.

Se encarga de obtener el máximo valor posible al mínimo costo. También es el responsable de la cartera de productos, conocida como pila de producto o \textit{Product Backlog}.  Por esta razón,  comprende las necesidades de los usuarios dentro del negocio.

En la ejecución de los proyectos ágiles, el \textit{Product Owner} normalmente posee las siguientes responsabilidades:

\begin{itemize}
\item Determinar los requisitos generales y actividades iniciales del proyecto.
\item Representar a los usuarios del producto.
\item Buscar y asegurar los recursos financieros que requiere el proyecto para iniciarse y desarrollarse.
\item Analizar la viabilidad del emprendimiento.
\item Garantizar que el producto se entregue.
\item Desarrollar y establecer los criterios para aceptar las historias de los usuarios.
\item Aprobar o negar los productos entregables.
\end{itemize}




